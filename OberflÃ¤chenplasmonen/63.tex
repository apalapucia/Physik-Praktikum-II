\RequirePackage{silence} % :-\
    \WarningFilter{scrreprt}{Usage of package `titlesec'}
    %\WarningFilter{scrreprt}{Activating an ugly workaround}
    \WarningFilter{titlesec}{Non standard sectioning command detected}
    \WarningFilter{latexrelease}{Package latexrelease Warning: Current format date selected, no patches applied}
\documentclass[ twoside,openright,titlepage,%1headlines,% letterpaper a4paper
                paper=a4,fontsize=11pt,%11pt,a4paper,%
                ngerman
                ]{scrartcl}

\usepackage{graphicx}
\usepackage{amsmath}
\usepackage[ngerman]{babel}
\usepackage[utf8]{inputenc}
\usepackage[T1]{fontenc}
\numberwithin{equation}{section}
%\usepackage{color}
%\usepackage{enumitem}
\usepackage{icomma}
%%\usepackage{titlesec}
%\usepackage{tikz,tkz-euclide}
%\usetkzobj{all}
%\usepackage{adjustbox}
\usepackage{multirow}
\usepackage{upgreek}
\usepackage{url}
\usepackage{wrapfig}
\usepackage{subcaption}
\usepackage{booktabs} 
\usepackage{geometry}
\begin{document}
\begin{titlepage}
    % if you want the titlepage to be centered, uncomment and fine-tune the line below (KOMA classes environment)
    \begin{addmargin}[-1cm]{-3cm}
    \begin{center}
        \large

        \hfill


        \begingroup
            \spacedallcaps{} \\ \bigskip
        \endgroup

	\spacedlowsmallcaps{David Hoffmann, Marius Tochtermann} \\
	Gruppe: D-05

        \vfill


        Assistent: \\ \medskip
        %\myDegree \\
        %\myFaculty \\
	Fortgeschrittenenpraktikum \\
        Universität Stuttgart\\ \bigskip

	.2019 

        \vfill

    \end{center}
  \end{addmargin}
\end{titlepage}
\tableofcontents
\section{Theory}
In order to describe surface plasmons the Maxwell equations are needed, as they are the theoretical fundament of electrodynamics. The four Maxwell equations lead to four boundary conditions at the transition area between two optical mediums, from which two apply to the electric field and two apply to the magnetic field. 

The first maxwell equation
\begin{align}
\vec{\nabla} \cdot \vec{D} = \vec{\nabla} \cdot \vec{E}\cdot \epsilon_0 \cdot \epsilon_\text{r} =  \rho_\text{frei} \label{eq:Maxwell1}
\end{align}
leads to the condition that the electric flux density needs to be steady in the direction perpendicular to the transition surface, as the density of electrical charges $\rho_\text{frei}$ equals zero inside the transition area. Due to the fact that the $\epsilon_\text{r}$ of the two media are not the same, the electric field is not steady.

Like the first Maxwell equation describes the electric flux density, the second Maxwell equation is about the magnetic flux density. Thus that no magnetic monopoles exist, the right side of the second Maxwell equation
\begin{align}
\vec{\nabla} \cdot \vec{B} = \vec{\nabla} \cdot \vec{H}\cdot \mu_0 \cdot \mu_\text{r} =  0 \label{eq:Maxwell2}
\end{align}
always equals zero. Like the electric flux density, the magnetic flux density needs to be steady in the direction perpendicular to the transit area. In most cases, the magnetic permeability is close to one. As a result the magnetic field can be assumed as steady too.

The third Maxwell equation 
\begin{align}
\vec{\nabla} \times \vec{E} = -\frac{\partial \vec{B}}{\partial t} \label{eq:Maxwell3}
\end{align}
can be used to describe the electric field parallel to the transition area. In consequence of the second Maxwell equation the magnetic flux density is steady. This leads, coupled with the third Maxwell equation, to a steadiness of the electric field parallel to the transition area.

The last Maxwell equation 
\begin{align}
\vec{\nabla} \times \vec{H} = \vec{j_\text{frei}} + \frac{\partial \vec{D}}{\partial t} \label{eq:Maxwell4}
\end{align}
is about the magnetic field. The absence of free charges leads to an absence of a current density ($\overrightarrow{j_\text{frei}} = 0$). With the steadiness of the electric flux density resulting from the first Maxwell equation, the magnetic field parallel to the transit area is steady too.

In absence of free electrical charges, the third and fourth Maxwell equation can be combined to the wave equation:
\begin{align}
\vec{\nabla}^2 \vec{E} = \epsilon_0 \epsilon_\text{r} \mu_0 \mu_\text{r} \frac{\partial^2}{\partial t^2} \vec{E} \label{eq:Waveequation}
\end{align} 
The wave equation can be solved with the approach of a plane wave. With linear combinations of plane waves, every other wave form can be build. A plane electromagnetical wave is described as
\begin{align}
\vec{E} = \vec{E_0} exp(\vec{k}\vec{r} - \omega t)
\end{align}
The approach for the magnetic field is identical. With those approaches the Maxwell equations can be written as:
\begin{align}
\vec{k} \cdot \vec{D} &= 0 \label{eq:Maxwell1alt} \\
\vec{k} \cdot \vec{B} &= 0 \label{eq:Maxwell2alt} \\
\vec{k} \times \vec{E} &= \omega \vec{B} \label{eq:Maxwell3alt} \\
\vec{k} \times \vec{H} &= -\omega \epsilon_0 \epsilon_r \vec{E} \label{eq:Maxwell4alt}
\end{align}
For further considerations the dispersion relation is very important. The dispersion relation describes the correlation between the absolut amount of the wavevector $k$ and the angular velocity $\omega$. The dispersion relation can be used to calculate the phase velocity $\frac{\omega}{k}$ and the group velocity $\frac{d\omega}{dk}$. 

In order to obtain the dispersion relation the cross product of the wave vector $k$ and equation \ref{eq:Maxwell3alt} is calculated. The result is
\begin{align}
\vec{k}^2 = \epsilon_\text{r} \frac{\omega^2}{c^2} \label{eq:Dispersionrelation}
\end{align}
It is important to mention that $\epsilon_\text{r}$ is dependent on $\omega$ and $c=\frac{1}{\sqrt{\epsilon_0 \cdot \mu_0}}$.

As surface plasmons are a phenomenon occuring in the transition area between to mediums, the transition area will be in focus now. The plane between the materials shall be identical with the y-z-plane. Without loss of generality, the wave vector of the incoming light shall be in the x-z-plane. The light shall be p-polarized which means that the electric field does not have a y-component. With this construction the x-component will behave in another way than the z-component. As shown above, the x-component features a steady electric flux density ($\epsilon_1 E_\text{z1} = \epsilon_2 E_\text{z2}$), while the z-component has a steady electric field ($E_\text{x1} = E_\text{x2}$). The wave vector is orthogonal to the electric field, due to the construction that the incoming light has no electric field or wave vector in the y-direction, the wave vector in x-direction is proportional to the dielectric factor while the wave vector in z-direction is the same in both media. Combined with the dispersion relation, the following connections can be build:
\begin{align}
k_\text{z}^2 &= \frac{\omega^2}{c^2} \frac{\epsilon_1 \epsilon_2}{\epsilon_1 + \epsilon_2} \label{eq:WavevectorZ} \\
k_\text{i}^2 &= \frac{\omega^2}{c^2} \frac{\epsilon_\text{i}^2}{\epsilon_1 + \epsilon_2} \label{eq:WavevectorI} \ \ \ \ \  i=1,2
\end{align}
In order to obtain surface plasmons an evanescent wave on both sides of the transition area is needed, this means that the electric field is decreasing its amount exponentially with an increasing distance to the transit area. To get a evanescent field $k_\text{i}^2$ needs to be smaller than zero. Surface plasmons are propagating along the transit area, this means that $k_\text{z}^2$ needs to be larger than zero ($k_\text{y}^2$ is zero due to the geometric construction). 

A closer look at the equations \ref{eq:WavevectorZ} and \ref{eq:WavevectorI} reveals, that the both dielectric functions need to have opposite signs. What seems to be impossible in the first place can be explained with the Drude model, which applies to media with free electrons such as metals. 

In the Drude model the dielectric function is not a real but a complex function. The Drude model is based on diffential equation with three forces. The inertia force of the electron, a damping force (electrical resistance etc.) depending on the drift velocity of the electron and the force of the electric field of the incoming electromagnetic wave on the electron:
\begin{align}
m \frac{\partial v}{\partial t} + m \frac{v_\text{d}}{\tau} &= -e E \label{eq:Drude1} \\
m \frac{d^2 r}{dt^2} + m \Gamma \frac{dr}{dt} &= -e E_0 exp(-i \omega t) \label{eq:Drude2} \ \ \ \ \ \Gamma = \frac{1}{\tau}
\end{align}
The solution of equation \ref{eq:Drude2} is an oscillating function:
\begin{align}
r(t) = r_0 exp(-i \omega t) = \frac{e E_0}{m(\omega(\omega - i \Gamma))} exp(-i \omega t) \label{eq:DrudeSolution}
\end{align}

The dielectric function is dependent on the polarizability of the medium. The polarizibility can be calculated as the Polarisation divided by the electric field:
\begin{align}
P_0 &= -n \cdot e \cdot r_0 = \frac{-n \cdot e^2 \cdot E_0}{m(\omega(\omega - i \Gamma))} \label{eq:Polarisation} \\
\epsilon (\omega) &= 1 + \frac{P_0}{\epsilon_0 \cdot E_0} = 1 - \frac{n \cdot e^2}{\epsilon_0 \cdot m(\omega(\omega - i \Gamma))} = 1 - \frac{\omega_\text{p}^2}{\omega(\omega - i \Gamma)} \label{eq:DielectricFunction} \\
\text{mit} \ \ \omega_\text{p}& = \sqrt{\frac{n \cdot e^2}{\epsilon_0 \cdot m}} \label{eq:PlasmaFrequency}
\end{align}

\end{document}
